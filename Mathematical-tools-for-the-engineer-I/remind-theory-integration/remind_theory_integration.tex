\documentclass[12pt]{article}

% package
%\usepackage{cmap}
\usepackage[utf8]{inputenc}
\usepackage[T1]{fontenc}
\usepackage[francais]{babel}
\usepackage{lmodern}
\usepackage{amsfonts}
\usepackage{amsmath}

% My Commands
%% ensemble nombre IN, IR, ...
\let\big\mathbb
\newcommand{\N}{\ensuremath{\big{N}}}
\newcommand{\R}{\ensuremath{\big{R}}}
\newcommand{\C}{\ensuremath{\big{C}}}
\newcommand{\Rn}[1]{\ensuremath{\big{R}^{#1}}}

\begin{document}
    \begin{center}
        \begin{LARGE}
        \underline{\textbf{Rappels sur la théorie de l'intégration}}\\
        \end{LARGE}
    \end{center}
	
	\section{Ensembles mesurables et mesure de Lebesgue}
		\subsection*{Définition}
		Un pavé P dans \Rn{d} est un produit cartésienne de d intervalles de \R \, bornés (ouvert, fermé, semi-ouvert ou semi-fermé).
		\begin{center}
		$ P = \prod^{d}_{i=1}\left]a_{i}\, ; b_{i} \right[ $ où $ a_{i} \leq b_{i}$, des nombres réels, $i = 1, ..., d$
		\end{center}
		
		\subsection*{Volume du pavé}
		\noindent \begin{align*}
		V 	&= \mid P\mid \\
			&= (b_{1} - a_{1})(b_{2} - a_{2})...(b_{d} - a_{d})
		\end{align*}
		
		\subsection*{Definition}
		Une union de pavé est dite disjoint si les intérieurs de pavé sont disjoints.
		
		\subsection*{Remarque}

\end{document}