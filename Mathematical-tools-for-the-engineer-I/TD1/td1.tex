\documentclass[12pt]{article}

% package
\usepackage{cmap}
\usepackage[utf8]{inputenc}
\usepackage[T1]{fontenc}
\usepackage{lmodern}
\usepackage{geometry}
\geometry{
top=2cm,
bottom=3cm,
left=3cm,
right=3cm
}
\usepackage{amsfonts} % this package permit to display lR, C and more


\begin{document}
\begin{center}
{\LARGE TD I Outils Mathématiques pour l'ingénieur}
\end{center}


\begin{center}
\underline{\textbf{Exercice 1}}
\end{center}
Donner le développement en série de Laurent des fonctions suivantes et préciser dans quelle partie 
de $ \mathbb{C} $ elles sont variables.
\\
\par\noindent 1) $ f(z) = \frac{1}{(z+2)(z-1)} $ autour de 0, de -1 et -2 \\
2) $ f(z) = \frac{z}{(z^{2}-1)} $ autour de 0, de 2 et 1 \\
3) $ f(z) = \frac{1}{(z-1)^{2}(z-3)} $ dans $ 0 < \mid z-1\mid < 2 $ \\

\par\noindent
\begin{center}
\underline{\textbf{Exercice 2}}
\end{center}
Calculer les intégrales suivantes:
\par\noindent
\begin{center}
\begin{tabular}{c c}
$ I_{1} = \displaystyle{\int_{\mid z\mid = 2}}\tan(z)dz $ &
$ I_{2} = \displaystyle{\int_{\mid z\mid = 2}}\frac{\exp(z)}{z^{4} + 5z^{3}} dz $
\\
 & 
\\
$ I_{3} = \displaystyle{\int_{\mid z-4\mid = \frac{1}{2}}}\frac{1}{z^{4}-1}dz $ &
$ I_{4} = \displaystyle{\int_{\mid z\mid = 2}}\frac{z^{99}\exp(\frac{1}{z})}{z^{100} + 1} dz $
\\
 & 
\\
$ I_{5} = \displaystyle{\int_{\mid z-i\mid = \frac{3}{2}}}\frac{\exp(\frac{1}{z^{2}})}{z^{2}+1} dz $ &
$ I_{6} = \displaystyle{\int_{\mid z\mid = 2}}\frac{\sin(\frac{1}{z})}{z-1}dz $
\end{tabular}
\end{center}

\par\noindent
\begin{center}
\underline{\textbf{Applications des résidus}}
\end{center}
Calculer:
\par\noindent
\begin{center}
\begin{tabular}{c c}
$ I_{1} = \displaystyle{\int_{-\infty}^{+\infty}}\frac{1}{x^{2}+1} dx $ &
$ I_{2} = \displaystyle{\int_{0}^{2\pi}}\frac{1}{x^{2}+1} dz $
\\
 & 
\\
$ I_{3} = \displaystyle{\int_{0}^{2\pi}}\frac{\cos(\alpha)}{5+\cos(\alpha)}d\alpha $ &
$ I_{4} = \displaystyle{\int_{-\infty}^{+\infty}}\frac{1}{x^{4}+1} dx $
\\
 & 
\\
$ I_{5} = \displaystyle{\int_{-\infty}^{+\infty}}\frac{\sin(x)}{x+i} dx $ &
$ I_{6} = \displaystyle{\int_{0}^{+\infty}}\frac{\sin(x)}{x} dx $
\end{tabular}
\end{center}

\par\noindent
\begin{tabular}{c c}
Calculer: &
$ \displaystyle{\sum_{n=-\infty}^{+\infty}}\frac{1}{n^{2}+a^{2}} $
\end{tabular}

\end{document}